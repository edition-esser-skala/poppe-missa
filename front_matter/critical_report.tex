\documentclass[shorttitlesize=55]{ees}

\shorttitle{Missa S: Antonii P}

\begin{document}

\eesTitlePage

\eesCriticalReport{
  – & –   & org  & \B1 only comprises ripieno bass parts (i.e., vlne and fag).
                   Thus, org (solo) sections and bass figures are only
                   available from \B2. \\
  1 & 59  & B    & 2nd to 4th \quarterNote\ in \B1 and \B2: d4–c4–B4 \\
    & 63f & org  & upper voice in treble clef missing in \B2 \\
  \midrule
  2 & 1   & vl 1 & 1st \quarterNote\ in \B1: a″16–e″16–e″8 \\
    & 18  & vla  & 4th \eighthNote\ in \B1 and \B2: g′8 \\
    & 21  & vla  & 4th \eighthNote\ in \B1: g′16–g′16 \\
    & 33  & vl 2 & 2nd \eighthNote\ in \B1 and \B2: d″8 \\
    & 34  & vla  & 6th \eighthNote\ in \B1 and \B2: c′8 \\
    & 34  & T    & 4th \quarterNote\ in \B1 and \B2: b8–a16 \\
    & 36f & ob 1 & 4th \quarterNote\ and subsequent bar in \B1:
                   a″8–\quaverRest\ and \wholeNoteRest \\
    & 47  & A    & 1st \eighthNoteDotted\ in \B1 and \B2: a′8. \\
    & 48  & vla  & 7th \eighthNote\ in \B1 and \B2: a′8 \\
    & 54  & org  & upper voice in \B2: e″8–\crotchetRest \\
    & 70  & ob 1 & 6th \eighthNote\ in \B1: b′16–b′16 \\
  \midrule
  3 & 1–17 & –   & In contrast to \B1, one bar in \B2 comprises six half notes,
                   a peculiarity which is retained in this edition. \\
    & 4   & vl 2 & 3rd \halfNote\ in \B1 and \B2: c′′′4.–c′′′8 \\
    & 7   & A    & 4th \halfNoteDotted\ in \B1 and \B2: e′2. \\
    & 17  & vla  & 2nd/3rd \halfNote\ in \B1 and \B2: a′2–\sharp g′2 \\
    & 25  & org  & bar almost illegible in \B2 \\
    & 38  & T    & 2nd \quarterNote\ in \B1 and \B2: d′4 \\
    & 41f & org  & bars almost illegible in \B2 \\
    & 60  & vl 1 & 2nd \eighthNote\ in \B1 and \B2: b′16–d″16 \\
    & 60  & vl 2 & 4th \quarterNote\ in \B1 and \B2: b′8–\sharp g′8 \\
    & 63  & vl 1 & 6th \eighthNote\ in \B1: b″8 \\
    & 63  & T    & 1st \quarterNote\ in \B1 and \B2: e′8–e′8 \\
    & 66  & ob   & 8th \sixteenthNote\ in \B1: a′16 \\
    & 67  & org  & upper voice in \B2: e″8–\crotchetRest \\
    & 68  & vl 1 & 2nd \halfNote\ in \B2: a″8–a″8–b″8–b″8 \\
    & 70  & vl 2 & 6th \eighthNote\ in \B1: b′16–b′16 \\
  \midrule
  4 & 4   & vl 2 & 7th \eighthNote\ in \B1 and \B2: \sharp g′8 \\
    & 4   & ob, S & 2nd/3rd \quarterNote\ in \B1 and \B2: f″2 \\
    & 6   & vl 1 & 7th \eighthNote\ in \B1: b″8 \\
    & 10  & vl 1 & 3rd \quarterNote\ in \B1 and \B2: e″4 \\
    & 14  & vl 1 & last \sixteenthNote\ in \B1 and \B2: \sharp g″16 \\
    & 20  & vla  & 2nd \quarterNote\ in \B1 and \B2: e′8–e′8 \\
    & 20  & A    & 1st \halfNote\ in \B1 and \B2: \sharp f′4.–e′8 \\
    & 21  & vla  & 6th/7th \eighthNote\ in \B1 and \B2: b8–e′8 \\
    & 28–47 & –  & The \textit{Benedictus} most likely has been written
                   by Jan Dismas Zelenka (it appears in his own hand).
                   In \B1 and \B2, vl and vla are written in bass clef
                   and one octave lower. \\
  \midrule
  5 & 7   & vl 1 & 2nd \quarterNote\ in \B1 and \B2: c″8–b′8 \\
    & 7   & vla  & 2nd \quarterNote\ in \B1: d8–e8 \\
    & 8   & A    & This rhythm appears both in \B1 and \B2. \\
    & 8   & T    & 3rd \quarterNote\ in \B1 and \B2: c′4 \\
    & 12–32 & –  & The \textit{Dona nobis} is indicated in all instrumental
                   parts of \B1 and in \B2 as “Dona Nobis Come Kyrie Secondo”
                   or similar. Here, the fugue subject has been emended
                   in bars 12f (ob, S, org), 14f (vla, T), 15 (org), 17 (org),
                   22f (org), 23f (vla, T), and 24 (org) to ensure a common
                   rhythm and text. \\
    & 18  & B    & 2nd to 4th \quarterNote\ in \B1: d4–c4–B4 \\
    & 22f & org  & upper voice in treble clef missing in \B2 \\
}

\eesToc{}

\eesScore

\end{document}
